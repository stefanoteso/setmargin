%%%% ijcai16.tex

\typeout{Constructive Preference Elicitation by Setwise Max-margin Learning}

\documentclass{article}
\usepackage{ijcai16}
\usepackage{times}
%\usepackage{latexsym}

\RequirePackage{amsmath,amssymb,amsfonts,amsthm}
\usepackage{algorithm}
\usepackage{algpseudocode}

% ---------------------------------------------------------------------------

\algrenewcommand{\algorithmicrequire}{\textbf{Input:}}
\algrenewcommand{\algorithmicensure}{\textbf{Output:}}

\renewcommand\[{\begin{equation}}
\renewcommand\]{\end{equation}}

% Bold math symbols
\newcommand{\bbA}{\mathbb{A}}
\newcommand{\bbB}{\mathbb{B}}
\newcommand{\bbC}{\mathbb{C}}
\newcommand{\bbD}{\mathbb{D}}
\newcommand{\bbE}{\mathbb{E}}
\newcommand{\bbF}{\mathbb{F}}
\newcommand{\bbG}{\mathbb{G}}
\newcommand{\bbH}{\mathbb{H}}
\newcommand{\bbI}{\mathbb{I}}
\newcommand{\bbJ}{\mathbb{J}}
\newcommand{\bbK}{\mathbb{K}}
\newcommand{\bbL}{\mathbb{L}}
\newcommand{\bbM}{\mathbb{M}}
\newcommand{\bbN}{\mathbb{N}}
\newcommand{\bbO}{\mathbb{O}}
\newcommand{\bbP}{\mathbb{P}}
\newcommand{\bbQ}{\mathbb{Q}}
\newcommand{\bbR}{\mathbb{R}}
\newcommand{\bbS}{\mathbb{S}}
\newcommand{\bbT}{\mathbb{T}}
\newcommand{\bbU}{\mathbb{U}}
\newcommand{\bbV}{\mathbb{V}}
\newcommand{\bbW}{\mathbb{W}}
\newcommand{\bbX}{\mathbb{X}}
\newcommand{\bbY}{\mathbb{Y}}
\newcommand{\bbZ}{\mathbb{Z}}

% Calligraphic math symbols
\newcommand{\calvar}[1]{\ensuremath{\mathcal{#1}}}
\newcommand{\calA}{\calvar{A}}
\newcommand{\calB}{\calvar{B}}
\newcommand{\calC}{\calvar{C}}
\newcommand{\calD}{\calvar{D}}
\newcommand{\calE}{\calvar{E}}
\newcommand{\calF}{\calvar{F}}
\newcommand{\calG}{\calvar{G}}
\newcommand{\calH}{\calvar{H}}
\newcommand{\calI}{\calvar{I}}
\newcommand{\calJ}{\calvar{J}}
\newcommand{\calK}{\calvar{K}}
\newcommand{\calL}{\calvar{L}}
\newcommand{\calM}{\calvar{M}}
\newcommand{\calN}{\calvar{N}}
\newcommand{\calO}{\calvar{O}}
\newcommand{\calP}{\calvar{P}}
\newcommand{\calQ}{\calvar{Q}}
\newcommand{\calR}{\calvar{R}}
\newcommand{\calS}{\calvar{S}}
\newcommand{\calT}{\calvar{T}}
\newcommand{\calU}{\calvar{U}}
\newcommand{\calV}{\calvar{V}}
\newcommand{\calW}{\calvar{W}}
\newcommand{\calX}{\calvar{X}}
\newcommand{\calY}{\calvar{Y}}
\newcommand{\calZ}{\calvar{Z}}

% Vectors
\newcommand{\vecvar}[1]{\ensuremath{\boldsymbol{#1}}}
\newcommand{\va}{\vecvar{a}}
\newcommand{\vb}{\vecvar{b}}
\newcommand{\vc}{\vecvar{c}}
\newcommand{\vd}{\vecvar{d}}
\newcommand{\ve}{\vecvar{e}}
\newcommand{\vf}{\vecvar{f}}
\newcommand{\vg}{\vecvar{g}}
\newcommand{\vh}{\vecvar{h}}
\newcommand{\vi}{\vecvar{i}}
\newcommand{\vj}{\vecvar{j}}
\newcommand{\vk}{\vecvar{k}}
\newcommand{\vl}{\vecvar{l}}
\newcommand{\vm}{\vecvar{m}}
\newcommand{\vn}{\vecvar{n}}
\newcommand{\vo}{\vecvar{o}}
\newcommand{\vp}{\vecvar{p}}
\newcommand{\vq}{\vecvar{q}}
\newcommand{\vr}{\vecvar{r}}
\newcommand{\vs}{\vecvar{s}}
\newcommand{\vt}{\vecvar{t}}
\newcommand{\vu}{\vecvar{u}}
\newcommand{\vv}{\vecvar{v}}
\newcommand{\vw}{\vecvar{w}}
\newcommand{\vx}{\vecvar{x}}
\newcommand{\vy}{\vecvar{y}}
\newcommand{\vz}{\vecvar{z}}
\newcommand{\valpha}{\vecvar{\alpha}}
\newcommand{\veps}{\vecvar{\varepsilon}}
\newcommand{\vphi}{\vecvar{\varphi}}
\newcommand{\vpsi}{\vecvar{\psi}}

% Operators
\DeclareMathOperator*{\argmax}{argmax}
\DeclareMathOperator*{\argmin}{argmin}

% Debugging

\usepackage{color}
\newcommand{\andrea}[1]{{\bf \textcolor{blue}{{\fbox{Andrea:} #1}}}}
\newcommand{\stefano}[1]{{\bf \textcolor{green}{{\fbox{Stefano:} #1}}}}
\newcommand{\paolo}[1]{{\bf \textcolor{red}{{\fbox{Paolo:} #1}}}}

% ---------------------------------------------------------------------------

\title{Constructive Preference Elicitation by Setwise Max-margin Learning}
\author{ST and AP and PV}

\begin{document}

\maketitle

\begin{abstract}
WRITEME
\end{abstract}

\section{Introduction}

WRITEME

- importance of recommendation systems and preference elicitation

- existing approaches and their limitations

% constructive preference elicitation
In this paper we take a {\em constructive} view on preference
elicitation, enlarging its scope from the selection of items among a
set of candidates to the synthesis of entirely novel
instances. Instances are represented as combinations of basic elements
(e.g. the components of a laptop) subject to a set of constraints
(e.g. the laptop model determines the set of available CPUs). A
utility function is learned over the feature representation of an
instance, as customary in many preference elicitation approaches. The
recommendation is then made by solving a constrained optimization
problem in the space of feasible instances, guided by the learned
utility. \andrea{other constructive preference elicitation approaches?}

% setwise max-margin formulation to deal with uncertainty in user utility
Learning a user utility function requires setting a preference
elicitation strategy and dealing with uncertain and possibly
inconsistent user feedback. Bayesian preference elicitation approaches
deal with both problems by building a probability distribution on
candidate functions and asking queries maximizing informativeness
measures such as expected value of information (EVOI)~\cite{}. These
approaches are however computationally expensive and cannot scale to
fully constructive scenarios, as shown in our experimental results.
We take a space decomposition perspective and jointly learn a set of
weight vectors, each representing a candidate utility function,
maximizing diversity between the vectors and consistency with the
available feedback. These two conflicting objectives tend to generate
equally plausible alternative hypotheses for the unknown
utility. Preference elicitation works by combining weight vector
learning with instance generation, so that each iteration of the
algorithm produces two outcomes: a set of weight vectors and a set of
instances, each maximizing its score according to one of the weight
vectors. 


- experimental results

- paper structure


\section{Related Work}

WRITEME

\section{Setwise Max-margin Learning}

\paragraph{Notation.} We use boldface letters $\vx$ to indicate vectors,
capital letters $X$ for matrices, and $[n]$ as a shorthand for the set $\{1, \ldots, n\}$.

\paragraph{Non-linear Formulation.}

Let $k$ be set size and $n$ be the number of examples (answers).

{\footnotesize

\begin{align}
    \max_{M, W, X}
        & \;\; M - \frac{\alpha}{k |\calD|} \sum_{i=1}^k \| \veps^{(i)} \|_1 - \frac{\beta}{k} \sum_{i=1}^k \| \vw^{(i)} \|_1 + \frac{\gamma}{k} \sum_{i=1}^k \langle \vw^{(i)}, \vx^{(i)} \rangle
        \nonumber
    \\
    \text{s.t.}
        & \;\; \forall \; i \in [k] \nonumber, \forall \; h \in [n]
    \\
        & \;\; \qquad \langle \vw^{(i)}, \vy^{(h)}_+ - \vy^{(h)}_- \rangle \geq M - \varepsilon^{(i)}_h \nonumber
    \\
        & \;\; \forall \; i, j \in [k], i \neq j \nonumber
    \\
        & \;\; \qquad \langle \vw^{(i)}, \vx^{(i)} - \vx^{(j)} \rangle \geq M \label{cnst-ws-m}
    \\
        & \;\; \forall \; i \in [k] \nonumber
    \\
        & \;\; \qquad \vw^\bot \leq \vw^{(i)} \leq \vw^\top \nonumber
    \\
        & \;\; \forall \; i \in [k] \nonumber
    \\
        & \;\; \qquad \vx^i \in \calX_{\text{feasible}} \nonumber
    \\
        & \;\; \forall \; h \in 1, \ldots, n, \forall \; i \in [k] \nonumber
    \\
        & \;\; \qquad \varepsilon^{(i)}_h \geq 0 \nonumber
    \\
        & \;\; M \geq 0 \nonumber
\end{align}

}

\paragraph{MILP Formulation.}

WRITEME

\begin{algorithm}
\begin{algorithmic}[1]
    \Procedure{PESetMargin}{$Y, m; \calW, \calX, \valpha, T$}
        \State $\calA \gets \emptyset$
        \For{$t = 1, \ldots, T$}
            \State $\vw^*_1, \ldots, \vw^*_m, \vx^*_1, \ldots, \vx^*_m \gets \text{{\sc Solve}}(Y, \calA, m; \calW, \calX, \valpha)$
            \For{$\vx^*_i,\vx^*_j \in \{ \vx^*_1, \ldots, \vx^*_m \} \; \text{{\bf s.t.}} \; i < j$}
                \State $\calA \gets \calA \cup \text{{\sc QueryUser}}(\vx^*_i,\vx^*_j)$
            \EndFor
        \EndFor
        \State $\vw^*, \vx^* \gets \text{{\sc Solve}}(Y, \calA, 1; \calW, \calX, \valpha)$
        \State ${\bf return}\; \vw^*, \vx^*$
    \EndProcedure
\end{algorithmic}
\end{algorithm}

\section{Experiments}

WRITEME

\section{Conclusion}

WRITEME

\section*{Acknowledgments}

WRITEME

\bibliographystyle{named}
\bibliography{ijcai16}

\end{document}
